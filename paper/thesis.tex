\documentclass[a4paper]{report}

\usepackage{fontspec}

\usepackage[round]{natbib}
\renewcommand{\bibname}{References}
\bibliographystyle{abbrvnat}


\begin{document}

\title{IPA alignment}
\author{Pavel Sofroniev}
\maketitle

\begin{abstract}
	This paper compares various algorithms for aligning IPA sequences in the context of computational historical linguistics.
\end{abstract}


\chapter{Introduction}


\chapter{Background}


\chapter{Methodology}

\section{Vectors}

This section describes the different methods of obtaining phoneme vector representations that have been used in the paper.

\subsection{One-hot encoding}

Under one-hot encoding, given a vocabulary of N distinct phonemes, each phoneme would be represented as a distinct binary vector of size N,
such that exactly one of its dimensions has value 1 and all its other dimensions have value 0.
In such a vector space, all vectors have length 1 and any two non-identical vectors are orthogonal.

One-hot encoding is a simple method, both conceptually and computationally, but it cannot be very useful for producing distance measures
because under its model each phoneme is equidistant from all the others.
For the purposes of this study one-hot vector representations are used as a baseline for comparing other methods.


\subsection{PHOIBLE features}

PHOIBLE Online is an ongoing project aiming to compile a comprehensive tertiary database of the world languages' phonological inventories.
At the time of writing the database contains phonological inventories for 1672 distinct languages making use of 2160 distinct IPA segments \citep{2014_Moran_al}.

As part of the project the developers are also maintaining a table of phonological features,
effectively mapping each IPA segment encountered in the database to a unique ternary feature vector
(feature values indicate either the presence, absence, or non-applicability of the respective feature).
The feature table includes 39 distinct features and is based on research by \citet{2009_Bruce} and \citet{2011_Moisik_al}.

The PHOIBLE feature vectors comprise phoneme representations grounded in phonology and linguistics theory
that could be readily used for a variety of computational endeavours.


\subsection{phon2vec}

Word2vec comprises two closely related model architectures for computing vector representations of words, first introduced by \citet{2013_Mikolov_al}.
As this study concerns phonemes rather than words, we call the method phon2vec.

Unlike the vectors obtained from one-hot encodings or PHOIBLE's feature matrix, phon2vec vectors are the output of the model being trained on data.
The training data that we use is the set of all transcriptions from the NorthEuraLex,
a comprehensive lexicostatistical database that provides IPA-encoded lexical data for languages of, primarily but not exclusively, Northern Eurasia \citep{2017_Dellert_Jäger}.
At the time of writing the database covers 1016 concepts from 107 languages, resulting in 121614 IPA transcriptions.
The latter are tokenised using ipatok, which is covered somewhere else.

The word2vec/phon2vec model is a parametric one, and the resulting vector representations can vary widely depending on the parameters' values.


\subsection{Neural network embeddings}




\chapter{Evaluation}


\chapter{Conclusion}


\bibliography{references}


\end{document}
