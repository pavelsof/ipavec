\documentclass[a4paper]{report}


\begin{document}

\title{IPA alignment}
\author{Pavel Sofroniev}
\maketitle

\begin{abstract}
	This paper compares various algorithms for aligning IPA sequences in the context of computational historical linguistics.
\end{abstract}


\chapter{Introduction}


\chapter{Background}


\chapter{Methodology}

\section{Vectors}

This section describes the different methods of obtaining phoneme vector representations that have been used in the paper.

\subsection{One-hot encoding}

Under one-hot encoding, given a vocabulary of N distinct phonemes, each phoneme would be represented as a distinct binary vector of size N,
such that exactly one of its dimensions has value 1 and all its other dimensions have value 0.
In such a vector space, all vectors have length 1 and any two non-identical vectors are orthogonal.

One-hot encoding is a simple method, both conceptually and computationally, but it cannot be very useful for producing distances measures
because under its model each phoneme is equidistant from all the others.
For the purposes of this study one-hot vector representations are used as a baseline for comparing other methods.


\subsection{PHOIBLE features}


\subsection{phon2vec}


\subsection{Neural network embeddings}


\chapter{Results}


\chapter{Conclusion}


\end{document}
